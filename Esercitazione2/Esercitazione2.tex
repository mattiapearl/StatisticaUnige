% Options for packages loaded elsewhere
\PassOptionsToPackage{unicode}{hyperref}
\PassOptionsToPackage{hyphens}{url}
%
\documentclass[
]{article}
\usepackage{amsmath,amssymb}
\usepackage{lmodern}
\usepackage{iftex}
\ifPDFTeX
  \usepackage[T1]{fontenc}
  \usepackage[utf8]{inputenc}
  \usepackage{textcomp} % provide euro and other symbols
\else % if luatex or xetex
  \usepackage{unicode-math}
  \defaultfontfeatures{Scale=MatchLowercase}
  \defaultfontfeatures[\rmfamily]{Ligatures=TeX,Scale=1}
\fi
% Use upquote if available, for straight quotes in verbatim environments
\IfFileExists{upquote.sty}{\usepackage{upquote}}{}
\IfFileExists{microtype.sty}{% use microtype if available
  \usepackage[]{microtype}
  \UseMicrotypeSet[protrusion]{basicmath} % disable protrusion for tt fonts
}{}
\makeatletter
\@ifundefined{KOMAClassName}{% if non-KOMA class
  \IfFileExists{parskip.sty}{%
    \usepackage{parskip}
  }{% else
    \setlength{\parindent}{0pt}
    \setlength{\parskip}{6pt plus 2pt minus 1pt}}
}{% if KOMA class
  \KOMAoptions{parskip=half}}
\makeatother
\usepackage{xcolor}
\IfFileExists{xurl.sty}{\usepackage{xurl}}{} % add URL line breaks if available
\IfFileExists{bookmark.sty}{\usepackage{bookmark}}{\usepackage{hyperref}}
\hypersetup{
  pdftitle={Esercitazione 2},
  pdfauthor={Bastreghi Ranalli Torriglia},
  hidelinks,
  pdfcreator={LaTeX via pandoc}}
\urlstyle{same} % disable monospaced font for URLs
\usepackage[margin=1in]{geometry}
\usepackage{graphicx}
\makeatletter
\def\maxwidth{\ifdim\Gin@nat@width>\linewidth\linewidth\else\Gin@nat@width\fi}
\def\maxheight{\ifdim\Gin@nat@height>\textheight\textheight\else\Gin@nat@height\fi}
\makeatother
% Scale images if necessary, so that they will not overflow the page
% margins by default, and it is still possible to overwrite the defaults
% using explicit options in \includegraphics[width, height, ...]{}
\setkeys{Gin}{width=\maxwidth,height=\maxheight,keepaspectratio}
% Set default figure placement to htbp
\makeatletter
\def\fps@figure{htbp}
\makeatother
\setlength{\emergencystretch}{3em} % prevent overfull lines
\providecommand{\tightlist}{%
  \setlength{\itemsep}{0pt}\setlength{\parskip}{0pt}}
\setcounter{secnumdepth}{-\maxdimen} % remove section numbering
\ifLuaTeX
  \usepackage{selnolig}  % disable illegal ligatures
\fi

\title{Esercitazione 2}
\author{Bastreghi Ranalli Torriglia}
\date{2022-03-24}

\begin{document}
\maketitle

\hypertarget{esercizio-1}{%
\section{Esercizio 1}\label{esercizio-1}}

\hypertarget{introduzione}{%
\subsection{Introduzione}\label{introduzione}}

\hypertarget{frequenza-degli-incontri-con-gli-amici-e-fascia-di-etuxe0}{%
\subsubsection{Frequenza degli incontri con gli amici e fascia di
età}\label{frequenza-degli-incontri-con-gli-amici-e-fascia-di-etuxe0}}

Vengono anlaizzate delle rilevazioni fatte durante un indagine
\textbf{ISTAT} del 2010: ``Aspetti della vita quotidiana''. In
particolare in questo paragrafo \emph{(esercizio 1)} si studia il
rapporto tra frequenza degli incontri con gli amici e la fascia di età
\includegraphics{Esercitazione2_files/figure-latex/unnamed-chunk-2-1.pdf}
\includegraphics{Esercitazione2_files/figure-latex/unnamed-chunk-2-2.pdf}
\includegraphics{Esercitazione2_files/figure-latex/unnamed-chunk-2-3.pdf}
\includegraphics{Esercitazione2_files/figure-latex/unnamed-chunk-2-4.pdf}
\includegraphics{Esercitazione2_files/figure-latex/unnamed-chunk-2-5.pdf}

\hypertarget{commento-sulle-frequenze-relative-congiunte-e-sulle-frequenze-marginali}.\\
Restringendo leggermente i ``margini di osservazione'', si può notare
come già le prime due colonne costituiscano circa il \emph{50\%} delle
osservazioni, suggerendo che nella popolazione analizzata questa sia
l'opzione più comune. Prima di consultare i profili marginali riga
(relativi alle fasce d'età), si può già notare come la popolazione tra
\textbf{25 e i 54 anni} sia quella con più osservazioni.\\
Ciò implica che in casi come la quantità di incontri giornalieri con
amici, in cui si nota una preponderanza della classe \textbf{6-14 anni},
poi evidenziata nei profili colonna, non si noti la massima frequenza
marginale colonna poichè la classe più frequente sopra citata non ne
prende attivamente parte (\emph{come fa in tutti i casi da più volte a
settimana fino a qualche volta al mese}).

Sembra che la classe più comune di età, tenda ad avere dei dati alquanto
ben distribuiti nelle prime 4 categorie osservate di quantità di
uscite/incontri con amici, il che può suggerire che altre osservazioni
potrebbero essere necessarie per le classi con meno rilevazioni.

\begin{quote}
\^{}1 costituita dalla somma delle frequenze relative congiunte, quindi
dall'altezza della colonna. Si osserva questa congruenza nel barplot
della frequenza marginale colonna
\end{quote}

\includegraphics{Esercitazione2_files/figure-latex/unnamed-chunk-3-1.pdf}
\includegraphics{Esercitazione2_files/figure-latex/unnamed-chunk-3-2.pdf}
\includegraphics{Esercitazione2_files/figure-latex/unnamed-chunk-3-3.pdf}

\hypertarget{esercizio-2}{%
\section{Esercizio 2}\label{esercizio-2}}

\hypertarget{introduzione-1}{%
\subsection{Introduzione}\label{introduzione-1}}

\hypertarget{luogo-consumazione-pasto-e-titolo-di-studio}{%
\subsubsection{Luogo consumazione pasto e titolo di
studio}\label{luogo-consumazione-pasto-e-titolo-di-studio}}

Si studiano i dati relativi ad un indagine ISTAT che correla il luogo di
consumazione del pasto e il titolo di studio.

\includegraphics{Esercitazione2_files/figure-latex/unnamed-chunk-4-1.pdf}
\includegraphics{Esercitazione2_files/figure-latex/unnamed-chunk-4-2.pdf}
\includegraphics{Esercitazione2_files/figure-latex/unnamed-chunk-4-3.pdf}
\includegraphics{Esercitazione2_files/figure-latex/unnamed-chunk-4-4.pdf}
\includegraphics{Esercitazione2_files/figure-latex/unnamed-chunk-4-5.pdf}

\hypertarget{studio-delle-frequenze-relative-congiunte-e-frequenze-marginali}{%
\subsection{Studio delle frequenze relative congiunte e frequenze
marginali}\label{studio-delle-frequenze-relative-congiunte-e-frequenze-marginali}}

Prima di studiare i dati è obbligatorio notare come le rilevazioni di
persone che mangiano a \textbf{casa} siano l'80 percento del totale.
Quindi come sicuramente renderanno poco efficace l'osservazione non
correlata alle deviazioni dei profili riga e colonna, come avveniva
nello studio precedente con la classe di età tra i 25 e i 54 anni.\\
Analogamente si può osservare come tra le rilevazioni meno del 10\%
abbia una laurea come titolo di studio. Quando si studieranno i relativi
profili colonna sarà quindi necessario rapportare la dimensione del dato
ottenuto con la frequenza totale dei laureati, cosa che verrà
evidenziata con lo studio delle deviazioni.

\includegraphics{Esercitazione2_files/figure-latex/unnamed-chunk-5-1.pdf}

\includegraphics{Esercitazione2_files/figure-latex/unnamed-chunk-6-1.pdf}
\includegraphics{Esercitazione2_files/figure-latex/unnamed-chunk-6-2.pdf}

\end{document}
